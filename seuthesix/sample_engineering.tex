\documentclass[algorithmlist, figurelist,tablelist, nomlist,engineering]{seuthesix}
\let\cleardoublepage\clearpage % 去除空白页
\renewcommand{\algorithmicrequire}{\textbf{输入:}}  
\renewcommand{\algorithmicensure}{\textbf{输出:}}  

\begin{document}
\categorynumber{000} % 分类采用《中国图书资料分类法》
\UDC{000}            %《国际十进分类法UDC》的类号
\secretlevel{公开}    %学位论文密级分为"公开"、"内部"、"秘密"和"机密"四种
\studentid{191771}   %学号要完整,前面的零不能省略。
\title{意图驱动的网络服务模型适配和转译方法研究}{}{Research on Intent-Driven Network Service Model Adaptation and Translation Method}{}
\author{张强}{Qiang Zhang}
\advisor{董永强}{副教授}{Yongqiang Dong}{Associate Professor}
\degreetype{工学硕士}{Master of engineering} % 详细学位名称
\major{计算机技术}
\submajor{网络智能化}
\defenddate{\today}
\authorizedate{\today}
\committeechair{}
\reviewer{}{}
\department{东南大学计算机科学与工程学院}{School of Computer science and engineering}
\seuthesisthanks{本课题的研究获xx资助}
\makebigcover
\makecover
\begin{abstract}{意图}

\end{abstract}

\begin{englishabstract}{Intent}
	
\end{englishabstract}

\setnomname{术语与符号约定}
\tableofcontents
\listofothers

\mainmatter

\chapter{绪论}
本章首先介绍了本文工作的研究背景和意义,指出当前网络发展趋势不断朝着自动化和智能化的方向演进,以及基于意图的网络对于当前网络架构的重要意义,进而体现了研究意图驱动的网络服务的重要性和意义所在;接着对本文的研究目标和研究内容进行了阐述;最后概要地列出了本文组织结构。\par
\section{研究背景与意义}

\section{研究目标与内容}
\subsection{研究目标}

\subsection{研究内容}

\section{全文组织结构}



\chapter{研究现状}

\section{基于意图的网络概述}
\section{L3VPN概述}
\section{网络服务架构研究现状}
\section{研究现状总结}
\section{本章小结}
本章介绍了相关工作。为后续章节的内容打下了基础。

\chapter{用户意图的适配}
\section{意图的表现和定义方法的研究}
\section{意图YANG模型到服务YANG的映射}
\section{服务YANG模型到网络YANG模型的适配}
\section{本章小结}

\chapter{异构NETCONF/YANG消息的转译}

\section{CTC算法}

\subsection{NETCONF和YANG消息的解析方法}

\begin{algorithm}[h]
	\caption{NETCONF报文解析}  
	\begin{algorithmic}[1] %每行显示行号  
		\REQUIRE $XML_{input}$
		\ENSURE 操作列表$op\_list $
		\STATE $op\_list$ <-- [\quad]
		\STATE op <-- null
		\FOR{node \textbf{in} $XML_{input}$}
		\IF{node \textbf{has} operation}
		\IF{op == null}
		\STATE op <-- operation
		\STATE path <-- $PATH{root\_to\_node}$
		\ELSE
		\STATE $next\_op$ <-- operation
		\STATE data <-- ${DATA{op\_to\_next\_op}}$
		\STATE $op\_list{\cdot}{push\_back}\{op,path,data\}$
		\STATE op <-- $next\_op$
		\STATE path <-- $PATH{root\_to\_node}$
		\ENDIF
		\ENDIF
		\ENDFOR
		\STATE\RETURN{$op\_list$}
	\end{algorithmic}  
\end{algorithm} 


\subsection{NETCONF消息操作处理的方案-模拟计算}


\subsection{控制器配置翻译}


\subsection{基于设备端当前配置的比较}

\section{Mediator转换框架的设计和实现}

\subsection{重要组件的分析实现}

\section{本章小结}

\chapter{总结与展望}
本章对全文工作进行了回顾和总结。
\section{全文工作总结}
\section{未来工作展望}

\acknowledgement
感谢每一个给予帮助的人。

\bibliographystyle{unsrt}
\bibliography{seuthesix}



\appendix

\chapter{欧几里得第二定理的证明}
\newtheorem{theorem}{定理}
\begin{theorem}
欧几里得第二定理(素数有无穷多个)\\
证明:用反证法。假设素数有有限个($N$个),记为$p_1,p_2,\dots,p_N$。则我们构造一个新的数,
\[
n=p_1p_2\dots p_N+1.
\]
由于$p_i,i=1,2,\dots,N$为素数,则一定不为$1$。于是对于任意的$p_i,i=1,2,\dots, N$,有
\[
p_i\not|n
\]
这表明,要么$n$本身为素数,要么$n$为合数,但是存在$p_1,p_2,\dots,p_N$之外的其他素数能够将$n$进行素因子分解。
不管哪种情况,都表明存在更多的素数。定理得证。\qed
\end{theorem}

\chapter{$\sqrt{2}$是无理数的证明}
\begin{theorem}
$\sqrt{2}$是无理数。\\
证明:用反证法。假设$\sqrt{2}$是有理数,则可表示为两个整数的商,即$\exists p,q, q\ne0$
\[
\sqrt{2}=\frac{p}{q}
\]
不失一般性,我们假设$p,q$是既约的,即$\gcd(p,q)=1$。对上式两边平方可得\\
\begin{align*}
2& =\frac{p^2}{q^2}\\
p^2&=2q^2.
\end{align*}
表明$p^2$为偶数,因此$p$为偶数,记$p=2m$。则
\begin{align*}
p^2&=4m^2=2q^2\\
q^2&=2m^2.
\end{align*}
表明$q$也为偶数,因此它们有公共因子$2$。这与它们既约的假设矛盾。定理得证。\qed
\end{theorem}

\resume{作者攻读硕士学位期间的研究成果}


\end{document}